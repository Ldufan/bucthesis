\documentclass{buctart}
\begin{document}

\institute{机电工程学院}
\major{机电工程实验班}
\class{机实1601}
\studentid{2016000000}
\title{基于\LaTeX{}排版的北京化工大学本科\\毕业设计论文模板}
\titleen{BEIJING UNIVERSITY OF CHEMICAL TECHNOLOGY UNDERGRADUATE GRADUATION DESIGN THESIS TEMPLATE BASED ON LATEX TYPESETTING}
\author{独小凡}
\teacher{独\hspace{1em}凡}
\maketitle%生成封面
\frontmatter%诚信申明
\abstractzh{\par
	毕业设计(论文)是对学生进行全面训练的重要实践性教学环节,是高等学校实现本科培养目标的重要阶段。为保证本科生毕业设计(论文)质量,请参照以下规范撰写毕业设计(论文)。\par
	摘要应具有独立性和自含性,即不阅读论文的全文,就能获得必要的信息。摘要中有数据、结论,是一篇完整的短文,可以独立使用和引用。摘要的内容应包含与论文正文等同量的主要信息,供读者确定有无必要阅读全文,也可供二次文献(文摘等)采用。摘要一般应说明研究工作的目的、实验方法、结果和最终结论等,重点突出具有创新性的成果和新见解。\par
	中文摘要一般为300字左右,英文摘要为1500印刷符号左右,含中、英文摘要关键词。英文摘要应与中文摘要内容一致。\par}%中文摘要
	{毕业设计,学位论文,\LaTeX{}模板}%中文关键词
\newpage
\abstracten{\par
	Graduation design (thesis) is an important practical teaching link for comprehensive training of students, and an important stage for colleges and universities to achieve the goal of undergraduate training. In order to ensure the quality of graduation design (thesis) for undergraduates, please refer to the following specifications to write graduation design (thesis). \par
	The abstract should be independent and self-contained, that is, the necessary information can be obtained without reading the full text of the paper. The summary contains data and conclusions, and is a complete short article that can be used and quoted independently. The content of the abstract should contain the same amount of main information as the body of the paper, for readers to determine whether it is necessary to read the full text, or for secondary literature (abstract etc.) The abstract should generally explain the purpose of the research work, experimental methods, results and final conclusions, etc., focusing on innovative results and new insights.\par}%英文摘要
	{Graduation design, degree thesis, \LaTeX{} template}%英文关键词
\clearpage
\tableofcontents%目录
%\addcontentsline{toc}{section}{目录}
\clearpage

\symbolpage{
	\begin{table}[h]
		\raggedright
		\zihao{4}
		\renewcommand\arraystretch{1.5}
%		\resizebox{\textwidth}{!}{%
			\begin{tabular}{p{6em}l}
				\LaTeX{} & 基于\TeX{}的排版系统 \\ 
				BUCT     & 北京化工大学         \\ 
				awsl     & 表示X太可爱          \\ 
			\end{tabular}%
%		}
\end{table}}
\startmain
\section{绪论}
\subsection{研究背景}{\par
	学位论文一般应由以下几个主要部分构成,依次为:封面,诚信申明,中、英文题目,中、英文摘要,关键词,目录,符号和缩略词说明,论文正文,参考文献,附录,致谢\textsuperscript{\cite{zjsw}}。\par
	正文是学位论文的核心部分,一般由绪论、本论文、结论构成。包括理论分析、数据资料、计算方法、实验和测试方法、实验结果的分析和论证、个人的论点和研究成果、结论以及相关图表、照片和公式等部分。论文要求实事求是、论点明确、逻辑清楚、层次分明、文字流畅、数据真实、公式推导计算结果无误,其写作形式可因学科的特点不同而有所不同。\par}
\subsection{国内外现状}{
}
	\begin{figure}[htbp]
		\begin{center}
			\includegraphics[scale=0.8]{image/tom.jpeg}\\
			\caption{\zihao{5}小小Tom}
			\label{fig1-tom}
		\end{center}
	\end{figure}
	
\clearpage
\section{正文}{\par
	正文是学位论文的核心部分,一般由绪论、本论文、结论构成。包括理论分析、数据资料、计算方法、实验和测试方法、实验结果的分析和论证、个人的论点和研究成果、结论以及相关图表、照片和公式等部分。\par
	论文要求实事求是、论点明确、逻辑清楚、层次分明、文字流畅、数据真实、公式推导计算结果无误,其写作形式可因学科的特点不同而有所不同\textsuperscript{\cite{kocher99}}。文中若有与导师或他人共同研究的成果,必须明确标示;如果引用他人的结论,必须明确注明出处,并与参考文献引用号一致\cite{Krasnogor2004e}。}


% 参考文献
\clearpage
{\zihao{5}
\bibliographystyle{plain}
\addcontentsline{toc}{section}{参考文献}
\bibliography{refs}}

\acknowledgements{\par
	衷心感谢导师 xxx 教授对本人的精心指导。他们的言传身教将使我终生受益。\par
	感谢 xx 实验室主任 xx 教授,以及实验室全体老师和同学们的热情帮助和支持!本课题承蒙国家自然科学基金资助,特此致谢。感谢\LaTeX{}和\dufan{}的模板,帮我节省了不少时间。}
\appendix{\par
	附录一般作为学位论文主体的补充项目。主要包括:正文内过于冗长的公式推导;供读者阅读方便所需要的辅助性的数学工具或重复性数据图表;由于过分冗长而不宜放置在正文中的计算机程序清单;具有重要参考价值的资料;论文使用的缩写说明等。附录置于参考文献之后,其页码与正文连续编排。\par}
\addcontentsline{toc}{section}{附录}
\end{document}



