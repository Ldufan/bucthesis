\documentclass{bucthesis}
\begin{document}

\institute{机电工程学院}
\major{机电工程实验班}
\class{机实1601}
\studentid{2016000000}
\title{基于\LaTeX{}排版的北京化工大学本科\\毕业设计论文模板}
\titleen{BEIJING UNIVERSITY OF CHEMICAL TECHNOLOGY UNDERGRADUATE GRADUATION DESIGN THESIS TEMPLATE BASED ON LATEX TYPESETTING}
\author{独小凡}
\teacher{独\hspace{1em}凡}
\maketitle%生成封面
\frontmatter%诚信申明
\abstractzh{\par
	毕业设计(论文)是对学生进行全面训练的重要实践性教学环节,是高等学校实现本科培养目标的重要阶段。为保证本科生毕业设计(论文)质量,请参照以下规范撰写毕业设计(论文)。\par
	摘要是学位论文的内容不加注释和评论的简短陈述。置于诚信声明后。\par
	摘要应具有独立性和自含性,即不阅读论文的全文,就能获得必要的信息。摘要中有数据、结论,是一篇完整的短文,可以独立使用和引用。摘要的内容应包含与论文正文等同量的主要信息,供读者确定有无必要阅读全文,也可供二次文献(文摘等)采用。摘要一般应说明研究工作的目的、实验方法、结果和最终结论等,重点突出具有创新性的成果和新见解。\par
	中文摘要一般为300字左右,英文摘要为1500印刷符号左右,含中、英文摘要关键词。英文摘要应与中文摘要内容一致。除非无法变通的办法可用以外,摘要中不用图、表、化学结构式、非公知公用的符号和术语。\par}%中文摘要
	{毕业设计,学位论文,\LaTeX{}模板}%中文关键词
\newpage
\abstracten{\par
	Graduation design (thesis) is an important practical teaching link for comprehensive training of students, and an important stage for colleges and universities to achieve the goal of undergraduate training. In order to ensure the quality of graduation design (thesis) for undergraduates, please refer to the following specifications to write graduation design (thesis). \par
	The abstract should be independent and self-contained, that is, the necessary information can be obtained without reading the full text of the paper. The summary contains data and conclusions, and is a complete short article that can be used and quoted independently. The content of the abstract should contain the same amount of main information as the body of the paper, for readers to determine whether it is necessary to read the full text, or for secondary literature (abstract etc.) The abstract should generally explain the purpose of the research work, experimental methods, results and final conclusions, etc., focusing on innovative results and new insights.\par}%英文摘要
	{Graduation design, degree thesis, \LaTeX{} template}%英文关键词
\clearpage
\tableofcontents%目录
%\addcontentsline{toc}{section}{目录}
\clearpage

\symbolpage{
	\begin{table}[h]
		\raggedright
%		\centering
		\zihao{4}
		\renewcommand\arraystretch{1.5}
%		\resizebox{\textwidth}{!}{%
			\begin{tabular}{p{6em}l}
				\LaTeX{} & 基于\TeX{}的排版系统 \\ 
				BUCT     & 北京化工大学         \\ 
				awsl     & 表示XX太可爱,啊我死了     \\ 
			\end{tabular}%
%		}
\end{table}}
\startmain
\chapter{绪论}
\section{学位论文一般要求}{\par
	学位论文一般应由以下几个主要部分构成,依次为:封面,诚信申明,中、英文题目,中、英文摘要,关键词,目录,符号和缩略词说明,论文正文,参考文献,附录,致谢\textsuperscript{\cite{zjsw}}。学位论文应采用国家正式公布实施的简化汉字和国标计量单位。\par
	学位论文中采用的术语、符号、代号全文前后必须统一,并符合规范化的要求。论文中使用新的专业术语、缩略语、习惯用语,应加以注释。使用国外新的专业术语、缩略语,必须在译文后用圆括号注明原文。学位论文须用A4纸双面打印。学位论文稿纸四周应留足空白边缘,以便装订、复制和读者批注。建议页面的上下方分别留边25mm,左右侧分别留边27 mm。\par
	学位论文的插图、照片必须确保能复制或缩微。学位论文的页码,从“绪论”数起(包括绪论、正文、参考文献、附录、致谢等),用阿拉伯数字编连续码;中英文摘要、目录、符号和缩略词说明等页码用罗马数字单独编连续码。来华留学本科生,其论文撰写也可采用英语,但必须有相应的中文摘要和中文目录。\par	
	正文是学位论文的核心部分,一般由绪论、本论文、结论构成。包括理论分析、数据资料、计算方法、实验和测试方法、实验结果的分析和论证、个人的论点和研究成果、结论以及相关图表、照片和公式等部分。论文要求实事求是、论点明确、逻辑清楚、层次分明、文字流畅、数据真实、公式推导计算结果无误,其写作形式可因学科的特点不同而有所不同。\par}
\subsection{封面}{\par
	封面由学校统一印制,内容包括:班级、学号、毕业设计(论文)题目、专业、学生姓名、指导教师姓名。\textsuperscript{\cite{zjsw}}。\par}
\subsection{诚信申明}{\par
	要申明所撰写毕业设计(论文)及参考资料等真实可靠;如有不实之处,则按照学校有关规定接受处罚。\par}
\subsection{关键词}{\par
	关键词是为了文献标引而从学位论文中选取出来用以表示全文主题内容信息款目的单词或术语。关键词用显著的字符另起一行,排在摘要的左下方。关键词一般不超过3个。\par}
\subsection{目录}{\par
	学位论文应有目录,排在摘要之后。目录页每行均由标题名称和页码组成,包括引言(或前言),主要内容的篇、章、条、款、项序号和标题,小结,参考文献、附录等。\par}
\subsection{符号说明}{\par
	符号、标志、缩略语、首字母缩写、计量单位、名词、术语等的注释说明。置于目录之后。\par}
\section{图片实例}{\par
	如图\ref{fig2-bigbang}所示
	\begin{figure}[htb]
		\begin{center}
			\includegraphics[width=\linewidth]{image/BigBang.png}\\
			\caption{BigBang}
			\label{fig2-bigbang}
		\end{center}
	\end{figure}
\begin{figure}[htb]
	\begin{center}
		\includegraphics[scale=0.8]{image/tom.jpeg}\\
		\caption{小小Tom}
		\label{fig1-tom}
	\end{center}
\end{figure}
}
	
	
\clearpage
\chapter{正文}{\par
	正文是学位论文的核心部分,一般由绪论、本论文、结论构成。包括理论分析、数据资料、计算方法、实验和测试方法、实验结果的分析和论证、个人的论点和研究成果、结论以及相关图表、照片和公式等部分。\par
	论文要求实事求是、论点明确、逻辑清楚、层次分明、文字流畅、数据真实、公式推导计算结果无误,其写作形式可因学科的特点不同而有所不同\textsuperscript{\cite{kocher99}}。文中若有与导师或他人共同研究的成果,必须明确标示;如果引用他人的结论,必须明确注明出处,并与参考文献引用号一致\cite{Krasnogor2004e}。\par
	我校毕业设计(论文)按学科类别划分为理工类和文法经管类(含英语)两大类。 (1)理工类毕业设计(论文)主要分为工程设计、科学实验、软件开发、理论研究和综合等类型。\par
	(2)文法、经管类专业的论文可以是理论性论文、应用性论文、应用软件设计或调查报告。论文不能是一些文献资料简单地、机械地堆砌。论文要有足够的依据;论点与论据要一致,论据要充分支持论点;要有必要的数据资料,定性分析与定量分析相结合;理论、观点、概念表达要准确、清晰。\par}
\section{参考文献}{\par
	在学位论文中引用参考文献时,应在引出处的右上方用方括号标注阿拉伯数字编排的序号;参考文献的排列按照文中引用出现的顺序一般列在正文的末尾。
\par}
\section{注释}{\par 注释作为脚注在页下分散著录。\par}
\section{附录}{\par
	附录一般作为学位论文主体的补充项目。主要包括:正文内过于冗长的公式推导;供读者阅读方便所需要的辅助性的数学工具或重复性数据图表;由于过分冗长而不宜放置在正文中的计算机程序清单;具有重要参考价值的资料;论文使用的缩写说明等。附录置于参考文献之后,其页码与正文连续编排。
\par}
\section{致谢}{\par
	对于提供各类资助、指导和协助完成论文研究工作的单位及个人表示感谢。致谢应实事求是,真诚客观。\par}
\chapter{学位论文格式编排}
\section{学位论文的封面}{\par
	学位论文的封面到各学院教务处领取。论文题目用三号宋体加粗、其他信息用四号宋体加粗打印在封面规定的位置上。论文题目(包括副题和标点符号)不超过36个汉字。严格按照封面模板格式控制各部分的字体、字号。
\par}
\section{学位论文中文摘要}{\par
	(1)论文题目为三号黑体字,可以分成1或2行居中打印。(2)论文题目下空一行居中打印“摘要”二字(小三号黑体),两字间空一格(注:“一格”的标准为一个汉字,以下同)。(3)“摘要”二字下空一行,打印摘要内容(四号宋体)。段落按照“首行缩进”格式,每段开头空二格,标点符号占一格。(4)摘要内容后下空一行打印“关键词:”(四号黑体),其后为关键词(四号宋体)。关键词数量一般为3个,用“,”号分隔,句末不加标点。
\par}
\section{学位论文英文摘要}{\par
	(论文中的英文一律采用“Times New Roman”字体。论文英文题目全部采用大写字母,可分成1-3行居中打印。每行左右两边至少留五个字符空格。\par
	(1)英文题目下空三行居中打印“ABSTRACT”,再下空二行打印英文摘要内容,英文摘要与中文摘要相对应。
	(2)摘要内容每段开头留四个字符空格。
	(3)摘要内容后下空二行打印“KEY WORDS:”,其后关键词小写。
	\par}
\section{目录}{\par“目录”两字居中打印(三号黑体字),下空两行为章(四号黑体)、条(小四)、款(小四)、项(小四)及其开始页码。章、条、款、项层次代号如下:}
\section{标题}{\par
	每章的标题以三号黑体字居中打印;``章”下空两行为条的标题,以四号黑体字左起打印;``条”下空一行为``款”的标题, ``款”下空一行为``项”的标题,以小四号黑体字左起打印。换行后打印论文正文。\par}
\section{正文}{\par
	中文正文采用小四号宋体打印,英文正文采用小四号Times New Roman。行间距22磅,段前、段后均为0磅。\par}
\section{图}{\par
	图应有编号,建议分章依序编排。如图1-1,2-2,分别表示第一章第一张图,第二章第二张图。图应有图题,置于图的编号之后,图的编号和图题应置于图下方的居中位置。图中标注、图题采用中英文对照,其英文字体为五号Times New Roman,中文字体为五号宋体。\par}
\section{表}{\par
	表应有编号,建议分章依序编排。如表1-1,2-2,分别表示第一章第一张表,第二章第二张表。每张表应有表题,置于表的编号之后,表的编号和表题应置于表上方的居中位置。表中标注、表题采用中英文对照,其英文字体为五号Times New Roman,中文字体为五号宋体。\par}
\section{公式}{\par
	公式序号一律采用阿拉伯数字分章依序编排;如:式(2-13)、式(4-5),其标注应于该公式所在行的最右侧,公式与编号之间用“……”连接;公式书写方式应在文中相应位置另起一行居中横排,对于较长的公式只可在符号处(+、-、*、/、≤、≥等)转行。\par}
\section{参考文献}{\par
	按照参考文献在文中出现的顺序采用阿拉伯数字连续编号,在引出处的右上方用方括号标注阿拉伯数字编排的序号;参考文献的排列按照文中引用出现的顺序列在正文的末尾。引用多篇文献时,只须将各篇文献的序号在方括号内全部列出,各序号间用``,”;如遇连续序号,可标注起讫序号。\par}
% 参考文献
%\clearpage
{\zihao{5}
\bibliographystyle{plain}
\addcontentsline{toc}{chapter}{参考文献}
\bibliography{refs}}

\acknowledgements{\par
	衷心感谢导师 xxx 教授对本人的精心指导。他们的言传身教将使我终生受益。\par
	感谢 xx 实验室主任 xx 教授,以及实验室全体老师和同学们的热情帮助和支持!本课题承蒙国家自然科学基金资助,特此致谢。感谢\LaTeX{}和\dufan{}的模板,帮我节省了不少时间。}
\appendix{A}{\par
	附录一般作为学位论文主体的补充项目。主要包括:正文内过于冗长的公式推导;供读者阅读方便所需要的辅助性的数学工具或重复性数据图表;由于过分冗长而不宜放置在正文中的计算机程序清单;具有重要参考价值的资料;论文使用的缩写说明等。附录置于参考文献之后,其页码与正文连续编排。\par}
\addcontentsline{toc}{chapter}{附录}
\end{document}



