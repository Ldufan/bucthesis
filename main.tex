\documentclass{buctart}
\begin{document}
	
\class{机实1601}
\studentid{2016013288}
\title{无键相工况下往复压缩机振动冲击相位\\自动识别方法研究}
\titleen{RESEARCH ON AUTOMATIC IDENTIFICATION METHOD OF VIBRATION AND SHOCK PHASE OF RECIPROCATING COMPRESSOR WITHOUT KEYPHASOR}
\author{雷芙常}
\teacher{王\hspace{1em}瑶}
\maketitle%生成封面
\frontmatter%诚信申明
\abstractzh{\par
	往复式压缩机是石油化工等行业不可或缺的关键设备,其零部件多、结构复杂、故障率高。传感器是状态信号获取的主要工具,传感器性能的不稳定或者安装操作时的失误,往往会产生错误的监测信号,其后果十分严重。在往复式压缩机中键相信号用来对采集的状态信号进行整周期重采样,实现信号和压缩机活塞位置的精确匹配,一旦键相传感器出现故障将导致信号与压缩机状态无法匹配,进而无法进行其故障检测与诊断。因此监测诊断系统传感器的优化以及少传感器下的智能诊断技术的需求日益强烈。\par
	本文通过查阅相关文献及资料,运用哈希算法计算图像相似度的方法,先确定截取窗口大小即周期长度,进而确定起始相位最终实现无键相信号的整周期截取。将算法截取的整周期信号与键相截取的信号进行相似度计算,得到了理想的结果。在此基础上运用插值求包络的方法进行振动冲击相位的识别,得到了满意的效果,最终实现了无键相工况下往复式压缩机振动信号的整周期截取以及相位的自动识别。\par}%中文摘要
	{往复式压缩机,无键相,故障诊断}%中文关键词
\newpage
\abstracten{\par
	The reciprocating compressor is an indispensable key equipment in the petrochemical industry and other industries. It has many parts, complex structure and high failure rate. Sensors are the main tool for obtaining status signals. Unstable sensor performance or errors during installation and operation often produce erroneous monitoring signals, with serious consequences. In the reciprocating compressor, the key phase signal is used to resample the collected status signal over the entire period to achieve accurate matching of the signal and the compressor piston position. Once the key phase sensor fails, the signal will not match the compressor status, and then Its failure detection and diagnosis cannot be carried out. Therefore, there is an increasing demand for the optimization of sensors in monitoring and diagnosis systems and the intelligent diagnosis technology with fewer sensors.\par
	In this paper, by referring to relevant literature and data, and using the hash algorithm to calculate the image similarity, the size of the intercept window, that is, the period length, is determined first, and then the initial phase is finally determined to achieve the full period intercept of the keyless phase signal. By calculating the similarity between the signal intercepted by the algorithm and the signal intercepted by the key phase, an ideal result is obtained. On this basis, the method of interpolating and enveloping was used to identify the vibration and shock phases, and satisfactory results were obtained. Finally, the full-cycle interception of the vibration signal of the reciprocating compressor and the automatic identification of the phase were realized under the keyless phase condition}%英文摘要
	{Reciprocating compressor, no key phase, fault diagnosis}%英文关键词
\clearpage
\tableofcontents%目录
%\addcontentsline{toc}{section}{目录}
\clearpage
\symbols{符号和缩略词说明符号和缩略词说明符号和缩略词说明符号和缩略词说明}
\symbolpage
\startmain
\section{绪论}
\subsection{研究背景}{\par
	随着现代科学技术的迅速发展, 机械设备日益朝着高度自动化的方向发展,造成机械设备逐渐复杂且零、部件之间的联系更加紧密。一旦某一部分发生故障,往往会引起整台设备的瘫痪,而且频繁的故障和较长的检修时间常常造成巨大的经济损失和人员伤亡事故的发生\textsuperscript{\cite{zjsw}}。\par}
\subsubsection{背景}{
	人们对机械设备的可靠性、可用性、可维修性、经济性与安全性提出了越来越高的要求,现代工业生产中的设备系统比以往更注重效率和能耗\textsuperscript{\cite{kocher99}},且环保的要求越来越高\cite{Krasnogor2004e}。\par}
	$a^{2x+3} 	\oint_{C} x^3, dx + 4y^2, dy$\par
	$$\begin{cases}
	a_1x+b_1y+c_1z=d_1\\
	a_2x+b_2y+c_2z=d_2\\
	a_3x+b_3y+c_3z=d_3\\
	\end{cases}
	$$
	\begin{figure}[htbp]
		\begin{center}
			\includegraphics[scale=1.1]{image/tom.jpeg}\\
			\caption{\zihao{5}小小Tom}
%			\label{fig1-1 img1}
		\end{center}
	\end{figure}
	
\clearpage
\section{正文}


% 参考文献
\clearpage
{\zihao{5}
\bibliographystyle{plain}
\addcontentsline{toc}{section}{参考文献}
\bibliography{refs}}

\acknowledgements{\par
	衷心感谢导师 xxx 教授和物理系 xxx 副教授对本人的精心指导。他们的言传身教将使
	我终生受益。	
	在美国麻省理工学院化学系进行九个月的合作研究期间,承蒙 xxx 教授热心指导与帮助,不
	胜感激。\par
	感谢 xx 实验室主任 xx 教授,以及实验室全体老师和同学们的热情帮助和支
	持!本课题承蒙国家自然科学基金资助,特此致谢。	
	感谢  和 10,帮我节省了不少时间。}
\appendix{\par
	附录附录附录附录附录附录附录附录附录附录附录附录附录附录附录附录附录附录附录附录附录附录附录附录附录附录附录附录附录附录附录附录附录附录附录附录附录附录附录附录附录附录}

\end{document}



